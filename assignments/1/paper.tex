\documentclass[12pt]{article}

% This first part of the file is called the PREAMBLE. It includes
% customizations and command definitions. The preamble is everything
% between \documentclass and \begin{document}.

\usepackage[margin=1in]{geometry}  % set the margins to 1in on all sides
\usepackage{graphicx}              % to include figures
\usepackage{amsmath,bm}            % great math stuff
\usepackage{amsfonts}              % for blackboard bold, etc
\usepackage{amsthm}                % better theorem environments
\usepackage{listings}
\usepackage{bbm}
\usepackage{hyperref}
\usepackage{tikz}
\usepackage[]{algorithm2e}
\usepackage{parskip} 			   % no paragraph indentation

\usetikzlibrary{arrows,automata}


% various theorems, numbered by section

\newtheorem{thm}{Theorem}[section]
\newtheorem{lem}[thm]{Lemma}
\newtheorem{prop}[thm]{Proposition}
\newtheorem{cor}[thm]{Corollary}
\newtheorem{conj}[thm]{Conjecture}

\DeclareMathOperator{\id}{id}

\newcommand{\bd}[1]{\mathbf{#1}}  % for bolding symbols
\newcommand{\RR}{\mathbb{R}}      % for Real numbers
\newcommand{\ZZ}{\mathbb{Z}}      % for Integers
\newcommand{\col}[1]{\left[\begin{matrix} #1 \end{matrix} \right]}
\newcommand{\comb}[2]{\binom{#1^2 + #2^2}{#1+#2}}

\lstset{ % General setup for the package
    language={[LaTeX]TeX},
    basicstyle=\footnotesize\sffamily,
    tabsize=4,
    columns=fixed,
    keepspaces,
    commentstyle=\color{red},
    keywordstyle=\color{blue},
    xleftmargin=.1\textwidth,
    xrightmargin=.1\textwidth
}

\begin{document}

\nocite{*}


\title{Discrete Optimization \\
       Assignment 1}

\author{Mehdi Nadif \& Viktor Hansen}

\maketitle

\begin{abstract}
  This is the first weekly assignment for the Discrete Optimization course offered at The Department of Computer Science, Uni. Copenhagen.
\end{abstract}

\pagebreak

\section*{Theoretical part - formulation and lower bounds}
\subsection*{1.1}
\paragraph{Any feasible solution satisfies constraints:} Consider a Hamiltonian tour $p = \pi(1) \rightarrow \pi(2) \rightarrow \hdots \rightarrow \pi(n) \rightarrow \pi(1)$ of $G$ defined by a bijection $\pi : V=\left\{ 1, \hdots, n \right\} \rightarrow \left\{ 1, \hdots, n \right\}$. Let $E_p = \left\{ \left(\pi(i),\pi(i+1) \right) \; | \; 1 \leq i \leq n-1 \right\} \cup \left\{ \left(\pi(n),\pi(1) \right) \right\}$ denote the set of edges traversed by $p$. Then
\begin{align*}
&\sum_{i \in V, i \neq j} x_{ij} = \sum_{i \in V, i \neq j} \mathbbm{1}_{E_p}((i,j)) = 1 \; \; \; \forall j \in V \\
\mathrm{and} \; \; \;  &\sum_{j \in V, j \neq i} x_{ij} = \sum_{j \in V, j \neq i} \mathbbm{1}_{E_p}((i,j)) = 1 \; \; \; \forall i \in V
\end{align*}
since there is exactly one edge entering and leaving and each vertex. To see that $p$ satisfies the subtour constraint, consider any subset $S \subset V$ s.t. $2 \leq |S| \leq n-2$. In the worst case, 

\paragraph{Any solution satisfying constraints is feasible:}

\subsection*{1.2}

We may first note that 
$$
\sum_{i=0}^n\binom{n}{i} = 2^n
$$
Since $2^n$ is the total number of subsets of a set of size $n$. Since we are only looking at subsets of size between 2 and $n-2$, the number of subsets must be 
\begin{align*}
&2^n - \binom{n}{0} - \binom{n}{1} - \binom{n}{n-1} - \binom{n}{n} \\
&=2^n - 1 - n - n - 1 \\&= 2^n - 2n - 2 
\end{align*}
since there are as many constraints as subsets, this is also the number of constraints.

\subsection*{1.3}

The number of constraints is equivalent to the number of combinations of $i \in V, j \in V \backslash\{1\}$. As $|V| = n$, there must therefore be $n(n-1)$ constraints.

\subsection*{1.4}

Even though there are many more constraints in the subtour formulation, the variables that have to be optimized are only the $x_{ij}$. As the compact formulation have the additional $t_i, \ i = 1,...,V$ variables to optimize on, branch and bound will be on more variables. Therefore, the depth of the branching tree will be deeper, and it might take more time to get to a feasible solution which could have been used as an incumbent to prune with.\\

Similarly, one may note that bounding the compact formulation by an LP relaxation yields weak bounds as the $x_{ij}$ can have very low values without breaking constraints. This is observed in a slightly weaker formulation, where we first change the constraints so that
\begin{align*}
t_j & \geq t_i + 1 - n(1 - x_{ij}), \quad i\in V, j\in V\backslash \{1\}\\
\Leftrightarrow 1 - \frac{t_j - t_i - 1}{n} & \geq x_{ij}
\end{align*}

if we formulate every combination of $i\in V, j \in V\backslash {1}$ as the set $A$, we may create the weaker bound
\begin{align*}
\sum_{(i,j)\in A} 1 - \frac{t_j - t_i - 1}{n} & \geq \sum_{(i,j)\in A} x_{ij}\\
\Leftrightarrow |A| - \sum_{(i,j)\in A} \frac{t_j - t_i - 1}{n} & \geq \sum_{(i,j)\in A} x_{ij}\\
\Rightarrow 
\end{align*}

\subsection*{1.5}

Since $v_1$ is in a cycle, it will have two incident edges that are part of the cycle it is in. Let the additional edge $e_\text{add}$ be the second lowest cost edge adjacent to $v_1$, and add it to the tree to create a cycle. If we do not include $e_\text{add}$, we have a subset of edges of $G$ that form a tree. The optimal of such a tree is equivalent to a minimum spanning tree, which we will call $\mathcal{M}$.\\
Now, consider a Hamiltonian tour $\mathcal{H}$. From $\mathcal{H}$, we remove the most costly of the two edges incident to $v_1$, which we call $e_{\max}$. Removing this edge, we now have a tree, and therefore we can safely say that
$$
\text{cost}(\mathcal{H} ) - \text{cost}(e_{\max}) \geq \text{cost}(\mathcal{M})
$$
Since $\mathcal{M}$ is the tree with lowest possible cost. By definition, we must have that $e_{\max} \geq e_\text{add}$ since $e_\text{add}$ was the second lowest cost of all edges and $e_{\max}$ was chosen to be the most costly among two edges. Therefore
$$
\text{cost}(\mathcal{M}) + \text{cost}(e_\text{add}) \leq \text{cost}(\mathcal{M}) + \text{cost}(e_\text{max}) \leq \text{cost}(\mathcal{H})
$$


\section*{Implementation part - branch-and-bound}
\subsection*{2.1}

There are several upper bounds that can be implemented for this TSP problem, and the ones that we have attempted to implement for this problem were:
\begin{enumerate}
\item {\bf maximumSpanningTree}: First, construct a spanning tree of maximum cost using Kruskal, but chosing the maximum edge at each iteration. Then select the remaining most expensive edge to create a cycle. This should bound the optimal Hamiltonian cycle from above, from the proof in subsection 1.5.
\item {\bf two selecter}: For each $v\in V$, select the two most expensive edges adjacent to it (assure that edges that are forced to be included in the branch is chosen, and excluded are ignored). Take the sum of all these edges and divide by two to get an upper bound. The division is necessary, since every edge is in two terms of the summation.
\item {\bf Christofides algorithm}: a more complex algorithm, that 
\item {\bf 2-approximation}: Creates a minimum spanning tree of the graph, and then computes the size of an euler tour on this minimum spanning tree. Slightly weaker than Christofides with a worst case of twice the optimal solution.
\end{enumerate}

As can be seen in Fig. \ref{upperbounds}, the results of each of these heuristics have greatly varying results. Clearly the 2-approximation has the strongest bound on the root node, but it proved quite complex to implement this algorithm so that it worked with the branching, as assuring that some edge would be included in the euclidean tour was quite difficult. Also, it seems to be a bound that only gets worse as you decrease the domain of the graph. This is not the case for the {\bf maximumSpanningTree} or the {\bf two selector} which both get better at each branching since they always chose worst case for a graph.\\

Unfortunately, none of these upper bounds improved the number of nodes visited from only pruning by lower bounds that are higher than the currently best feasible solution. 
\begin{figure}
\begin{tabular}{|c|c|c|c|}
\hline
&maxSpan & 2-selecter & 2-approximation\\
\hline
Instance 1 & 31.59 & 31.08 & 8.90 \\
\hline
Instance 2 & 43.02 & 38.71 & 30.83\\
\hline
Instance 3 & 54.38 & 51.82 & 44.86\\
\hline
\end{tabular}
\caption{The upper bounds on the root node problems of the three test instances using three of the above mentioned four upper bound heuristics.\label{upperbounds}}
\end{figure}

\subsection*{2.2}

\subsection*{2.3}

\begin{figure}
\begin{tabular}{|c | c | c | c | c |}
\hline
& \multicolumn{2}{|c|}{Branch and bound} & \multicolumn{2}{|c|}{CPLEX} \\
\hline 
& Time & Nodes visited & Time & Nodes Visited\\
\hline
Instance 1 & & & & \\
\hline
Instance 2 & & & & \\
\hline
Instance 3 & & & & \\
\hline 
\end{tabular}
\end{figure}

\end{document}