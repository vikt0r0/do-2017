\documentclass[12pt]{article}

% This first part of the file is called the PREAMBLE. It includes
% customizations and command definitions. The preamble is everything
% between \documentclass and \begin{document}.

\usepackage[margin=1in]{geometry}  % set the margins to 1in on all sides
\usepackage{graphicx}              % to include figures
\usepackage{amsmath,bm}            % great math stuff
\usepackage{amsfonts}              % for blackboard bold, etc
\usepackage{amsthm}                % better theorem environments
\usepackage{listings}
\usepackage{bbm}
\usepackage{hyperref}
\usepackage{tikz}
\usepackage[]{algorithm2e}
\usepackage{parskip} 			   % no paragraph indentation

\usetikzlibrary{arrows,automata}


% various theorems, numbered by section

\newtheorem{thm}{Theorem}[section]
\newtheorem{lem}[thm]{Lemma}
\newtheorem{prop}[thm]{Proposition}
\newtheorem{cor}[thm]{Corollary}
\newtheorem{conj}[thm]{Conjecture}

\DeclareMathOperator{\id}{id}

\newcommand{\bd}[1]{\mathbf{#1}}  % for bolding symbols
\newcommand{\RR}{\mathbb{R}}      % for Real numbers
\newcommand{\ZZ}{\mathbb{Z}}      % for Integers
\newcommand{\col}[1]{\left[\begin{matrix} #1 \end{matrix} \right]}
\newcommand{\comb}[2]{\binom{#1^2 + #2^2}{#1+#2}}

\lstset{ % General setup for the package
    language={[LaTeX]TeX},
    basicstyle=\footnotesize\sffamily,
    tabsize=4,
    columns=fixed,
    keepspaces,
    commentstyle=\color{red},
    keywordstyle=\color{blue},
    xleftmargin=.1\textwidth,
    xrightmargin=.1\textwidth
}

\begin{document}

\nocite{*}


\title{Discrete Optimization \\
       Assignment 1}

\author{Mehdi Nadif \& Viktor Hansen}

\maketitle

\begin{abstract}
  This is the first weekly assignment for the Discrete Optimization course offered at The Department of Computer Science, Uni. Copenhagen.
\end{abstract}

\pagebreak

\section*{Theoretical part - formulation and lower bounds}
\subsection*{1.1}
\paragraph{Any feasible solution satisfies constraints:} Consider a Hamiltonian tour $p = \pi(1) \rightarrow \pi(2) \rightarrow \hdots \rightarrow \pi(n) \rightarrow \pi(1)$ of $G$ defined by a bijection $\pi : V=\left\{ 1, \hdots, n \right\} \rightarrow \left\{ 1, \hdots, n \right\}$. Let $E_p = \left\{ \left(\pi(i),\pi(i+1) \right) \; | \; 1 \leq i \leq n-1 \right\} \cup \left\{ \left(\pi(n),\pi(1) \right) \right\}$ denote the set of edges traversed by $p$. Then
\begin{align*}
&\sum_{i \in V, i \neq j} x_{ij} = \sum_{i \in V, i \neq j} \mathbbm{1}_{E_p}((i,j)) = 1 \; \; \; \forall j \in V \\
\mathrm{and} \; \; \;  &\sum_{j \in V, j \neq i} x_{ij} = \sum_{j \in V, j \neq i} \mathbbm{1}_{E_p}((i,j)) = 1 \; \; \; \forall i \in V
\end{align*}
since there is exactly one edge entering and leaving and each vertex. To see that $p$ satisfies the subtour constraint, consider any subset $S \subset V$ s.t. $2 \leq |S| \leq n-2$. Then for each such subset $S$,
\begin{align*}
\sum_{i,j \in S} x_{ij} = \sum_{i,j \in S} \mathbbm{1}_{E_p}((i,j)) \leq \sum_{i \in S} \left[ \mathbbm{1}_{E_p}((\pi(i),\pi(i+1))) \right] - 1 = \sum_{i \in S} \left( 1 \right) - 1 = |S| - 1
\end{align*}
This first inequality follows since in the worst case, $S$ forms a Hamiltonian path (i.e. subgraph of the Hamiltonian tour), and this has one edge less than the number of vertices.

\paragraph{Any solution satisfying constraints is feasible:}
To show the other direction, we need to show that an assignment satisfying the constraints correspond to a Hamiltonian tour of G, i.e. that
\begin{enumerate}
\item All vertices are visited exactly once (except for the first vertex)
\item There is only one cycle.
\end{enumerate}
Suppose $\mathbf{x} \in \left\{0,1\right\}^{n \times n}$ is an assignment satisfying the contraints. To show 1), we note that $x_{ij}=1$ iff. there is an edge going from vertex $v_{i}$ to $v_{j}$. For each vertex $v_{k}$ to be visited exactly once, we require that $\sum_{i \in V} x_{ik} = \sum_{j \in V} x_{kj} = 1$, $\forall k \in 1 \hdots n$ which is satisfied by assumption.

To show 2), assume for the sake of contradiction that the feasible solution consists of $k>1$ subtours $p_{1}, \hdots , p_{k}$. The sets of vertices in each such subtour, $p_{i}$, denoted by $L(p_{i})$, form a partition of $V$, so $2 \leq \left| L(p_{i}) \right| \leq n-2$, $1 \leq i \leq k$. Since any subtour $p_{i}$ is a cycle, it has exactly $|L(p_i)|$ edges, and hence $\sum_{i,j \in L(p_{i})} x_{ij} = |L(p_{i})|$ which contradicts the subtour constraint. Hence there can only one cycle.


\subsection*{1.2}

We may first note that 
$$
\sum_{i=0}^n\binom{n}{i} = 2^n
$$
Since $2^n$ is the total number of subsets of a set of size $n$. Since we are only looking at subsets of size between 2 and $n-2$, the number of subsets must be 
\begin{align*}
&2^n - \binom{n}{0} - \binom{n}{1} - \binom{n}{n-1} - \binom{n}{n} \\
&=2^n - 1 - n - n - 1 \\
&= 2^n - 2n - 2 
\end{align*}
since there are as many constraints as subsets, this is also the number of constraints.
\subsection*{1.3}
The number of constraints is equal to the number of combinations of $i \in V, j \in V \backslash\{1\}$. As $|V| = n$, there are $\left|V \times V \setminus \left\{1\right\} \right| = n(n-1)$ constraints.

\subsection*{1.4}


\subsection*{1.5} 
Consider a minimum-cost Hamiltonian tour $\mathcal{H}$, a minimum spanning tree $\mathcal{M}$ in $G$ and a leaf-node $v_1$ in $\mathcal{M}$. Since $v_1$ is on $\mathcal{H}$, it has exactly two incident edges $e_\text{min}$ and $e_{\text{max}}$. From $\mathcal{H}$, we remove the most costly of the two edges incident to $v_1$, denoted by $e_{\max}$. Removing this edge, we now have a tree, and therefore
$$
\text{cost}(\mathcal{M}) \leq \text{cost}(\mathcal{H}) - \text{cost}(e_{\max})
$$
since $\mathcal{M}$ is the tree with lowest possible cost. By definition, we must have that $e_{\max} \geq e_\text{min}$ since $e_\text{max}$ was the costlier among the two edges. Therefore
$$
\text{cost}(\mathcal{M}) + \text{cost}(e_\text{min}) \leq \text{cost}(\mathcal{M}) + \text{cost}(e_\text{max}) \leq \text{cost}(\mathcal{H})
$$
and hence the MST with the addition of cheapest edge to a leaf is a lower bound for $\text{cost}(\mathcal{H})$

\section*{Implementation part - branch-and-bound}
\subsection*{2.1}
\subsection*{2.2}
\subsection*{2.3}

\end{document}
