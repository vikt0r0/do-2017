\documentclass[12pt]{article}

% This first part of the file is called the PREAMBLE. It includes
% customizations and command definitions. The preamble is everything
% between \documentclass and \begin{document}.

\usepackage[margin=1in]{geometry}  % set the margins to 1in on all sides
\usepackage{graphicx}              % to include figures
\usepackage{amsmath,bm}            % great math stuff
\usepackage{amsfonts}              % for blackboard bold, etc
\usepackage{amsthm}                % better theorem environments
\usepackage{listings}
\usepackage{hyperref}
\usepackage{tikz}
\usepackage[]{algorithm2e}
\usepackage{parskip} 			   % no paragraph indentation

\usetikzlibrary{arrows,automata}


% various theorems, numbered by section

\newtheorem{thm}{Theorem}[section]
\newtheorem{lem}[thm]{Lemma}
\newtheorem{prop}[thm]{Proposition}
\newtheorem{cor}[thm]{Corollary}
\newtheorem{conj}[thm]{Conjecture}

\DeclareMathOperator{\id}{id}

\newcommand{\bd}[1]{\mathbf{#1}}  % for bolding symbols
\newcommand{\RR}{\mathbb{R}}      % for Real numbers
\newcommand{\ZZ}{\mathbb{Z}}      % for Integers
\newcommand{\col}[1]{\left[\begin{matrix} #1 \end{matrix} \right]}
\newcommand{\comb}[2]{\binom{#1^2 + #2^2}{#1+#2}}

\lstset{ % General setup for the package
    language={[LaTeX]TeX},
    basicstyle=\footnotesize\sffamily,
    tabsize=4,
    columns=fixed,
    keepspaces,
    commentstyle=\color{red},
    keywordstyle=\color{blue},
    xleftmargin=.1\textwidth,
    xrightmargin=.1\textwidth
}

\begin{document}

\nocite{*}


\title{Discrete Optimization \\
       Assignment 1}

\author{Mehdi Nadif \& Viktor Hansen}

\maketitle

\begin{abstract}
  This is the first weekly assignment for the Discrete Optimization course offered at The Department of Computer Science, Uni. Copenhagen.
\end{abstract}

\pagebreak

\section*{Theoretical part - formulation and lower bounds}
\subsection*{1.1}
\subsection*{1.2}

We may first note that 
$$
\sum_{i=0}^n\binom{n}{i} = 2^n
$$
Since $2^n$ is the total number of subsets of a set of size $n$. Since we are only looking at subsets of size between 2 and $n-2$, the number of subsets must be 
\begin{align*}
&2^n - \binom{n}{0} - \binom{n}{1} - \binom{n}{n-1} - \binom{n}{n} \\
&=2^n - 1 - n - n - 1 \\&= 2^n - 2n - 2 
\end{align*}
since there are as many constraints as subsets, this is also the number of constraints.
\subsection*{1.3}
The number of constraints is equivalent to the number of combinations of $i \in V, j \in V \backslash\{1\}$. As $|V| = n$, there must therefore be $n(n-1)$ constraints.
\subsection*{1.4}



\subsection*{1.5}

Since $v_1$ is in a cycle, it will have two incident edges that are part of the cycle it is in. Let the additional edge $e_\text{add}$ be the second lowest cost edge adjacent to $v_1$, and add it to the tree to create a cycle. If we do not include $e_\text{add}$, we have a subset of edges of $G$ that form a tree. The optimal of such a tree is equivalent to a minimum spanning tree, which we will call $\mathcal{M}$.\\
Now, consider a Hamiltonian tour $\mathcal{H}$. From $\mathcal{H}$, we remove the most costly of the two edges incident to $v_1$, which we call $e_{\max}$. Removing this edge, we now have a tree, and therefore we can safely say that
$$
\text{cost}(\mathcal{H} ) - \text{cost}(e_{\max}) \geq \text{cost}(\mathcal{M})
$$
Since $\mathcal{M}$ is the tree with lowest possible cost. By definition, we must have that $e_{\max} \geq e_\text{add}$ since $e_\text{add}$ was the second lowest cost of all edges and $e_{\max}$ was chosen to be the most costly among two edges. Therefore
$$
\text{cost}(\mathcal{M}) + \text{cost}(e_\text{add}) \leq \text{cost}(\mathcal{M}) + \text{cost}(e_\text{max}) \leq \text{cost}(\mathcal{H})
$$


\section*{Implementation part - branch-and-bound}
\subsection*{2.1}
\subsection*{2.2}
\subsection*{2.3}

\end{document}
