\documentclass[12pt]{article}

% This first part of the file is called the PREAMBLE. It includes
% customizations and command definitions. The preamble is everything
% between \documentclass and \begin{document}.

\usepackage[margin=1in]{geometry}  % set the margins to 1in on all sides
\usepackage{graphicx}              % to include figures
\usepackage{amsmath,bm}            % great math stuff
\usepackage{amsfonts}              % for blackboard bold, etc
\usepackage{amsthm}                % better theorem environments
\usepackage{listings}
\usepackage{bbm}
\usepackage{hyperref}
\usepackage{url}
\usepackage{tikz}
\usepackage[]{algorithm2e}
\usepackage{parskip} 			   % no paragraph indentation

\usetikzlibrary{arrows,automata}


% various theorems, numbered by section

\newtheorem{thm}{Theorem}[section]
\newtheorem{lem}[thm]{Lemma}
\newtheorem{prop}[thm]{Proposition}
\newtheorem{cor}[thm]{Corollary}
\newtheorem{conj}[thm]{Conjecture}

\DeclareMathOperator{\id}{id}

\newcommand{\bd}[1]{\mathbf{#1}}  % for bolding symbols
\newcommand{\RR}{\mathbb{R}}      % for Real numbers
\newcommand{\ZZ}{\mathbb{Z}}      % for Integers
\newcommand{\col}[1]{\left[\begin{matrix} #1 \end{matrix} \right]}
\newcommand{\comb}[2]{\binom{#1^2 + #2^2}{#1+#2}}

\lstset{ % General setup for the package
    language={[LaTeX]TeX},
    basicstyle=\footnotesize\sffamily,
    tabsize=4,
    columns=fixed,
    keepspaces,
    commentstyle=\color{red},
    keywordstyle=\color{blue},
    xleftmargin=.1\textwidth,
    xrightmargin=.1\textwidth
}

\begin{document}

\nocite{*}


\title{Discrete Optimization \\
       Assignment 2}

\author{Mehdi Nadif \& Viktor Hansen}

\maketitle

\begin{abstract}
  This is the second weekly assignment for the Discrete Optimization course offered at The Department of Computer Science, Uni. Copenhagen.
\end{abstract}

\pagebreak

\section*{Theoretical part}
Let $X_a = 1$ if $a$ is covered by $C$ and $X_a=0$ otherwise and let $X=\sum_{a \in V} X_a$ denote the total number of elements covered by $C$. We wish to bound the probability $Pr\left[ X \geq n/2 \right]$ using Chernoff Bound, since the bound yielded by Markov's inequality on this expression is too weak to say anything meaningful. $X$ is a sum of Bernoulli trials with parameters $p_a  \geq p \geq  1-\frac{1}{e}$. Using this bound for $p_a$ we can apply applying Chernoff bound for sum of Bernoulli variables \footnote{\url{https://en.wikipedia.org/wiki/Chernoff_bound\#Example}} and get:
\begin{align*}
\text{Pr}[X \geq n/2] \geq 1 - e^{-\frac{n}{2p}\left( p-\frac{1}{2} \right)^2} \geq 1 - e^{-\frac{n}{2-\frac{2}{e}}\left( 1-\frac{1}{e}-\frac{1}{2} \right)^2} \geq 1 - e^{-\frac{1}{2-\frac{2}{e}}\left( 1-\frac{1}{e}-\frac{1}{2} \right)^2} \approx 0.014
\end{align*}
Now, consider a rounded set-cover $C$ produced by the LP program. We wish to bound $\text{Pr}[\text{cost}(C) \leq c \cdot OPT ] = 1 - \text{Pr}[\text{cost}(C) > c \cdot OPT ] $. From Markov's inequality we get
\begin{align*}
\text{Pr}[\text{cost}(C) > c \cdot OPT ] \leq \text{Pr}[\text{cost}(C) \geq  c \cdot OPT_f ] = \text{Pr}[\text{cost}(C) \geq  c \cdot \text{E}[\text{cost}(C)] ] \leq 1/c
\end{align*}
So  $\text{Pr}[\text{cost}(C) \leq c \cdot OPT ] \geq 1-1/c$. Finally by applying union bound we get the following lower bound
\begin{align*}
\text{Pr}[X \geq n/2 \; \land \; c \cdot OPT \geq \text{cost}(C)] = & \; 1-\text{Pr}[X < n/2 \; \lor \; c \cdot OPT < \text{cost}(C)] \\
\geq &\;1 - (1 - 0.014) - (1-1/c) \\
= & \; 1 - 1 + 0.014 - 1 + 1/c\\
= & \; c-0.086 \\
= & \; \Omega(1)
\end{align*}
For the upper bound we have $\text{Pr}[X \geq n/2 \; \land \; c \cdot OPT \geq \text{cost}(C)]  \leq \min \left\{ 0.014, 1/c \right\} = O(1)$ showing that the event occurs with  probability $\Theta(1)$.

\pagebreak
\section*{Implementation part}
\subsection*{2.1 CPLEX}

\subsection*{2.2 Rounding}

\subsection*{2.3 Third Method}
The implemented metaheuristic uses a hill-climbing (valley descent) approach to determine a local minimum. The approach initialises a valid set cover and greedily removes the most costly set while maintaining feasibility until no more sets can be removed. Two approaches of producing initial feasible solutions were tested:
\begin{enumerate}
\item In which one set covering each vertex was picked uniformly at random.
\item In which all sets were picked initially.
\end{enumerate}
1) was made with the idea of adversarial situations in mind, in which input instances might foil the greedy approach used by the algorithm, however approach 2) produced much better results and was chosen instead. Our implementation thus lacks the ability to foil adversaries and 'escape' local minimums, which could be remedied by the introduction of random choices. Enlarging the neighborhood to occasionally include infeasible solutions would also seem worthwhile, however both of these strategies call for more sophisticated heuristics, e.g. simulated annealing. The results of the heuristic can be seen in Table \ref{times} - we were surprised how well the heuristic managed to find good covers despite its simplicity.
\begin{table}[!hbt]
\center
\begin{tabular}{|c | c | c | c | c | c |}
\hline
& \multicolumn{1}{|c|}{CPLEX} & \multicolumn{1}{|c|}{Simple Rounding (f/val)} & Randomized Rounding & \multicolumn{1}{|c|}{Descent} \\
\hline
scpa3 & 232 & (3, 313) & 397.6 & 256 \\
\hline
scpc3 & 243 & (4, 379) & 462.9 & 273 \\
\hline
scpnrf1 & 14 & (7, 25) & 49.3 & 20 \\
\hline
scpnrg5 & N/A & (5, 321) & 391.7 & 196 \\
\hline 
\end{tabular}
\caption{Reported costs for the set covers found by our implementions. Simple rounding is written with lowest integer $f$ with a feasible solution and the objective value next to it at that $f$. The randomized rounding results are the mean of 10 feasible solutions' objective values.
\label{times}}
\end{table}

\end{document}
